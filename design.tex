\section{Experimental design}
\label{sec:experimental-design}

Our experiments probe limits of accuracy of \zplus.
Our first two experiments consider what happens as our qualitative abstraction
is a more or less accurate reflection of reality.  It is relatively accurate when the
probabilities of events it treats as qualitatively distinct are far apart, and
less and less accurate as they come closer and closer together. \hide{This is similar
to the way big-$O$ analysis degrades as the constants get larger and larger.}
Our third experiment considers how MIFD degrades as its input sensors become
less and less precise.
Finally, we consider how well MIFD addresses base rate problems -- to what
extent, and under what circumstances fusing redundant sensors qualitatively will
increase accuracy.

In all of these experiments, we define sensor arrangements as well as the probabilities for attacks, confounding
benign events, sensor false positives and false negatives. We sample from
them to find ground truth and sensor report sets, run MIFD on the sensor
reports, and evaluate its accuracy.
These experiments are necessarily artificial, since they assume
omniscience about parameters which are even in principle inaccessible.
Note that while the models are generated artificially, they
have the same structure as models used in real deployments of the
Scyllarus system, although we have limited ourselves to more simple cases.

\hide{
Our experiments are based on \emph{settings} comprised of
%% a set of  \emph{locations} (simulating hosts),
attack and benign event \emph{prototypes}, plus
\emph{sensors} which simulate \idses, each
sensing some number of attacks and possibly also (inadvertently)
benign events.  Since there is no opportunity to perform 
fusion if events are not sensed, in our simulated configurations each attack
event will be sensed by some number of sensors, greater than 1.  We vary this
sensor-to-attack ratio in our experiments.}
% In this first set of experiments, we do not simulate networks as
% such, so all of the locations are treated as identical: the same set
% of events can occur everywhere, and the same set of sensors are
% deployed at each location.

% DONE: Can you give the reader a sense of the distinction between
% configuration, setting, and run? The notion of "run" is relatively clear, but
% not the other two.  Something like "For example, a configuration might specify
% how a set of attacks is generated, and the set of sensors. A particular
% setting would involve a particular set of possible attacks, a particular set
% of sensors, and a particular set of sensing relations between attacks and
% sensors."
%
% FIXME - Robert, in the marked-up PDF you suggested discussing
% locations, but earlier we'd decided to prune that discussion, since
% we didn't end up using different locations.
% Right, but here how will the reader know that we are going to sample from the
% same event multiple times?  The fact is that the whole notion of "run" as
% distinct from "location" is broken, so we must make the best of it that we
% can.  Please have a whack at this.
For each experiment we create a \emph{configuration},
a set of parameters specifying how a set of events and sensors, a
\emph{setting}, is generated by sampling random variables and
comparing to these parameters.  A
setting comprises a set of possible attacks, a set
of sensors, and a set of sensing relations between attacks and
sensors.
In each \emph{run} of a setting we sample %, for each location,
% some number of 
from the attack and benign events.
% These are
% Bernoulli random variables; we discuss setting their probabilities
% below.
In each experiment, for each configuration, we generated 1,000 settings
and conducted 100 runs per setting.
%
After sampling events, we sample from the sensors
%at the corresponding location,
according to their false positive and false negative probabilities.  
\hide{For each of
the $i$ events which the sensor might detect, we consider the probability
$\mathit{fn}$ of a false negative, so the probability of the sensor firing is $1
- \mathit{fn}^i$. If none of the sensed events occurred, the sensor will fire
according to its false positive probability.
}

\hide{
When measuring recall, we care only about trials where attacks
actually occur, but in some configurations the probability of an attack
is so small that the sample size required to meaningfully assess
recall with standard sampling would not be feasible.
Conversely, when measuring precision, false positives are of critical
importance, so samples without attacks are interesting.
Accordingly we sample separately for precision and recall, and
for recall we perform rejection sampling, rejecting any event samples
in which no attacks occur.}

Having generated the events and sensor reports for a run, we then use
the reports as
input to MIFD, and assess the resulting Bayes networks.  We extract
the set of attack event hypotheses that have been labeled as
\emph{likely} by MIFD.
We evaluate MIFD's performance in terms of \emph{precision} and \emph{recall},
comparing the set of events that MIFD considers likely with ground
truth generated for the run.
Recall is the percentage of actual attack events
which are labeled as likely. \hide{We compute recall as the fraction
\[\frac{\textrm{correct positives}}
       {\textrm{correct positives}+\textrm{false
           negatives}}\enspace.\] }
%
Precision is the percentage of attack events labeled as likely which 
actually occurred.\hide{
in fact part of the ground truth. We compute precision as the fraction
\[\frac{\textrm{correct positives}}
       {\textrm{correct positives}+\textrm{false positives}}\enspace.\]
%
We generally consider precision and recall aggregated over all runs
from a setting.}
Because of the high rate of reports and the low rate of events, precision and
recall must be in the high nineties, or the sensing system will not be usable.
We report the effect of our experiments on precision and recall in our
results section below.

\textbf{Configurations.}
\label{sec-5-1}
%% Explain how we generate settings, events, benign events, and sensor
%% reports.
% Generation of each experimental run of \mifd\ is governed by a number
% of experimental parameters.  
% Over any particular experiment, we will vary one or two parameters,
% and set others to values typical for an \ids.
In each of our experiments, we start with a configuration that reflects the
realities of IDS fusion, and whose parameter values are broadly consistent with
the simplifying assumptions of \zplus.
Configuration parameters may be either constants or parameterized random
variables.
We then vary some configuration parameter in ways which degrade
MIFD's performance. These experiments show us MIFD's performance under best-case conditions, and
how that performance degrades.
\begin{compactitem}
\item \numEventProtosDetected\ --- The total number of event types (both
  benign and attacks) to be
  detected by each sensor, a measure of the sensor's specificity. 
  Initially we use a random variable which returns 1 80\%
  of the time, and 2 otherwise.
\item \sensorToAttackRatio\ --- The number of sensors which should
  detect each attack event, initially 3.
\item \sensorOverlap\ --- The number of types of attack to be detected by each
  sensor, specified as a probability checked when deciding whether to
  associate an additional attack type with a sensor, initially 0.2.
\item \fpKappa\ --- The qualitative probability of
  false-positives for the sensors. We choose these to be 1
  in configurations where it takes 3 reports to rank a hypothesis likely, or 2
  where it takes only 2.
% 
%% Commenting out this bullet point --- not really relevant.
%
% \item $\kappa(\text{false-negative})$ --- MIFD does not currently reason about false negatives;
%  we take this probability to be whatever
%  corresponds to $\kappa = 2$ in the configuration.
%
\item \attackKappa\ and \benignKappa\ --- The qualitative
  probabilities of attack and benign events, initially
  2 and 1 respectively.

  %% This item is commented out in favor of the one below ---
  %% because we don't actually use \numBenignEvents in any experiment
  %% reported here.

  %% \item \numBenignEvents\ and \sensorToBenignRatio\ --- Exactly one
  %%   of these parameters should be given, and the other will be
  %%   derived. The former specifies a total number of benign events
  %%   which should be used in generated settings; the latter
  %%   specifies the number of sensors which should detect each benign
  %%   event.  We default to a constant value of 3 for
  %%   \sensorToBenignRatio.
\item \sensorToBenignRatio\ --- The number of sensors which should
  detect each benign event, initially 3.

\item \kappaTranslations\ --- Translations (into a constant or into a
  random variable) of qualitative probability values to real values in
  $[0,1]$. Initially we translate to constants, $\kappa(0)=0.5$,
  $\kappa(1)=0.01$, $\kappa(2)=0.001$.
  
% \item \numLocations --- Defaults to 1.

\item \numAttacks\ --- % Number of attack prototypes; 
  The number of attacks to take place, %at each location
  initially 4.


\end{compactitem}
%
% The values of \sensorToAttackRatio\ and the various \emph{kappa}s are
% interrelated. A sensible scenario should respect the orderings
% \begin{align}
% \end{align}
% or in other words,
% \begin{compactitem}
% \item A single sensor false positive is less surprising than an
%   attack event.
% \item A benign event is also less surprising than an attack event.
% \item An attack event is less surprising than false positives from all
%   of the sensors which detect that event.
% \end{compactitem}
%
%% (run-config ...)
%%  > (make-configuration-settings configuration)
%%  > (make-setting :configuration configuration)
%%  > (make-instance 'setting :configuration configuration ...)
%%    :after (initialize-instance ((obj setting)) ...)
%%             > (make-attack-events num-attacks attack-kappa)
%%                 ;; just the attack protos
%%    (setf (slot-value setting 'sensors) (make-sensors setting))
%%                 ;; sensor and benign protos
%%    (stratus-simulator:init-random-mifd-world setting num-locations
%%    ...)
%
%
\textbf{Settings.}
When creating a setting, we first consult
the \numAttacks\ parameter to determine the number of attack types.
Each type is associated with an
occurrence probability dictated by the 
\attackKappa parameter. The number of sensors is not fixed, but instead depends
on several configuration parameters. For each attack, we keep track of
the number of sensors which we must assign to detect it; this value is
initially set from \sensorToAttackRatio.  We create new sensors as
long as any attack requires assignment to an additional sensor. A new
sensor is initially assigned a total number of events which it will
detect from \numEventProtosDetected, and some set of attack
events. Each sensor is assigned at least one attack, plus
additional attacks up to its total number of events based on a 
Bernoulli random variable with parameter
\sensorOverlap.  After all sensor types are
created and assigned attacks, we assign benign events to the sensors
to reach their total number of events.
%% \begin{itemize}
%% \item If we are given \numBenignEvents, then we create that total
%%   number of benign event types, and distribute them among the
%%   sensors using a \sensorToBenignRatio\ which allows each sensor's
%%   total event count to be satisfied.
%% \item
%%   If we are given a \sensorToBenignRatio, then we create the
%%   necessary number of benign event types to satisfy both
%%   \sensorToBenignRatio\ and each sensor's total event count.
%% \end{itemize}
We create enough benign event types to satisfy
both \sensorToBenignRatio\ and each sensor's total event count, and
distribute them randomly among the sensors according to their total
event counts. 
% Finally we generate a network with the number of nodes
% determined by \numLocations.

%% (nondegenerate-simulate-step setting ...)
%%   > (simulate-step setting)
%%   > (events-for-location setting location)
%%     (reports-for-events setting location events)
% From the setting we can generate the ground truth aspect for an actual
% \mifd\ run: actual attack and benign events, plus the sensor reports
% which those events (or false positives) trigger. We proceed by
% considering each location of the network in turn.  We create the
% events which occur at each location by considering each attack and
% benign type. For each type, if a sample of a Bernoulli
% random variable whose success probability is the occurrence
% probability of the type is 1, then the ground truth aspect will
% contain a sensor report with that type and location.
% %
% For sensor reports at each location, we consider each sensor report
% type of the setting. If a type indicates that the sensor can
% detect one of the events which we have generated for that location,
% then we may have a report based on the events; otherwise we may have
% report because of a false positive.
% \begin{compactitem}
% \item If we have detectable events, then for each one we sample a
%   Bernoulli random variable whose success probability is the false
%   negative probability of the report type. If \emph{any} of these
%   samples are 0 --- that is, if for any event we do \emph{not} have a
%   false negative --- then the ground truth aspect will contain a
%   sensor report with that type and location.
% \item If there are no events at a location which could trigger a
%   report of that type, then we consult a Bernoulli random
%   variable whose success probability is the false positive probability
%   of the sensor report type; if it returns 1, then the ground
%   truth aspect will contain a sensor report with that type and
%   location.
% \end{compactitem}

\textbf{The experiments.}
% Our experiments probe the performance limits of our
% information fusion approach.  Our first two sets of tests examine how MIFD's
% performance degrades as events considered qualitatively different grow
% closer in likelihood.
% We then test how MIFD's performance degrades as the sensors it
% incorporates grow less and less discriminating.
% Finally, we examine how MIFD's sensor fusion helps base rate issues,
% and provide some analytic information.

\emph{Varying probabilities.}
Our first experiment examines how MIFD's performance degrades as
we progressively violate the assumption that different levels of
the stratified likelihood ranking qualitatively differ.
Specifically we consider a sequence of settings in which we assign
probabilities to each \tkappa,
always with
$P(\kappa=0)=0.5$ but varying the values for 1 and 2.
We test with \sensorToAttackRatio{}s of 2 and 3 to see how much corroboration is
necessary.  In the case of two sensors/attack we took \fpKappa=2, \benignKappa=2, \attackKappa=3
instead of the defaults above, in order to satisfy
Equation~\ref{eqn:coherence-ineq}.
We bring the probabilities corresponding to
the \tkappas closer and closer to see how MIFD's performance degrades.

\emph{Non point-value distributions.}
The above experiment still represents a considerable abstraction:  in general
the set of events to which we assign the same qualitative likelihood will not
all have the same probability.
Our second experiment examines how variation in the probabilities of the
events at the same qualitative likelihood affects the accuracy of qualitative
fusion.
To do so, we
define a second order probability distribution
for each \(\kappa\) ranking
using a \emph{beta distribution},
the Bayesian prior for Bernoulli
distributions.
The $\beta$ has two parameters, $\alpha$ and $\beta$.
The sum $\alpha+\beta$
corresponds to increasing the
``virtual sample size,'' and controls the variance.
% Beta distributions are appropriate for distributions over a finite
% domain such as the translations of these $\kappa$ values.
To assess the sensitivity of MIFD to variance in the probabilities,
we fix the mean values
of the distributions for $\kappa(0)$,
$\kappa(1)$ and $\kappa(2)$ at $0.5$, $0.01$ and $0.001$,
and increase the variance by decreasing the number of virtual samples.
% representing decreasing certainty in our parameter assignment.
% and vary the sample size $\nu=\alpha+\beta$, taking $\alpha=\mu\nu$,
% $\beta=(1-\mu)\nu$ for the translated distributions.  Higher sample
% sizes give smaller variance, and are thus closer to the single-value
% translations above; with smaller sample sizes the distributions will
% overlap, in the cases of our distributions for $\kappa(1)$ and
% $\kappa(2)$ with $\alpha<1$ and thus modes at $x=0$.

\emph{Sensor imprecision.}
We next explore how the performance of our techniques degrade as the
sensors encounter an increasing overlap of possible events in their
``field of view.'' 
For this experiment we fix the \sensorOverlap\ to be 0.9, and vary
\numEventProtosDetected. This value for \sensorOverlap is
significantly higher than in earlier runs, but we do not expect
performance in Setting~1 of this experiment to differ greatly from
Setting~1 of the earlier experiments because of the low initial values of
\numEventProtosDetected:  80\% of the time in this first run there will be only a single
event associated with a sensor, in which cases the \sensorOverlap\ is
not consulted at all.  With the higher
values from \numEventProtosDetected\ in subsequent settings, more
attacks will be associated with each sensor, increasing the opportunity
for \mifd\ to misdiagnose the true cause of sensor reports.

\emph{Base rate.}
Our fourth experiment examines how our performance
degrades
as the rate of true attacks goes down.
Such base rate problems, where even a high accuracy sensor can perform unacceptably
for very unlikely events, plague intrusion detection~\cite{Axelsson:1999:BFI:319709.319710}.
We examine these issues by reducing
the probability of generating attack events
(but not the probability of generating benign events). \hide{We do not actually change the
\attackKappa\ value used for the assessment step,
but instead introduce a
discrepancy between the model and the actual attack event probability.}



%%% Local Variables: 
%%% mode: latex
%%% TeX-master: "main"
%%% End: 

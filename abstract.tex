\begin{abstract}
We experimentally analyze the accuracy of the \zplus qualitative
probability scheme of
Goldszmidt and Pearl when used for diagnosis and information fusion.
The \longids
(\ids) fusion system
Scyllarus, and its successor MIFD, use $Z+$ to assess the likelihood of various cyber attack events based on
reports from \idses.
$Z+$ provides an order of magnitude approximation of
conventional probability, similar to the order of magnitude approximation of
computational complexity provided by big-O analysis.
Scyllarus accurately identifies attacks and
substantially reduces the false positives that are the bane of intrusion detection.
In the work described here, we experimentally analyze the performance of MIFD
in order to provide general conclusions about its behavior, complementing the
results from field tests.
Our experiments show that the qualitative probability scheme degrades gracefully
in precision and recall
as its order of magnitude approximation is a less and less accurate
representation of true distributions.
The system also degrades gracefully as its input sensors become
less discriminating.
Finally, we show that qualitatively fusing multiple IDSes successfully
addresses base rate issues in intrusion detection.
The interest of these results is not limited to intrusion detection: the method
used in our systems is a general abductive scheme, based on qualitative Bayes
networks, so the results are applicable to other information fusion and
diagnostic applications.
To the best of our knowledge, ours is the only experimental
investigation of the accuracy of $Z+$ as an approximation of conventional probability.
\end{abstract}


%%% Local Variables: 
%%% mode: latex
%%% TeX-master: "main"
%%% End: 

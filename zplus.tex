% \subsection{System Z+}
% \label{sec:zplus}

MIFD
follows Pearl's exhortation~\shortcite{Pearlbook} that the core of probability theory is
the patterns of inference it enables (explaining away, combining forward causal
inference and evidential reasoning, etc.), rather than precise numerical
calculations.  He recommends these patterns be used even when precise numbers are
not available.
%% TODO Restore for QR'15
%As outlined above,
MIFD treats
\ids fusion as an abductive problem, formalized using Bayes nets.
But precise values for the parameters of these Bayes nets
%% TODO Restore for QR'15
% --- probabilities of attacks, false positives, false negatives, etc. ---
are \emph{never} available to us: populations of
attacks and attackers change, different networks have little in common with each other, distributions are
non-stationary, etc.  Moreover these difficulties are not solvable via machine learning:
Sommer and Paxson~\shortcite{Sommer10outsidethe} discuss the challenges \idses pose to ML.

Goldszmidt and
Pearl's \zplus{}~\shortcite{Goldszmidt:96}
shares the
basic structure of normal probability theory but abstracts the actual
probabilities.
Events have a natural number rank \tkappa{}
corresponding to their degree of surprise (so a rank of one is more surprising
than zero). The semantics of this scheme comes from a set of probability
distributions in which the probabilities are polynomials over some infinitesimal
$\epsilon$. In this scheme, the \tkappa{} corresponds to the exponent
of the polynomial's leading term.
%% TODO Restore for QR'15
%As we mentioned earlier, 
\zplus is similar to the ``big-$O$'' method used
in analysis of algorithms.
With this semantics \zplus{}
provides a ``ladder'' of events of qualitatively different
orders of likelihood.
%% TODO Restore for QR'15
%Note that \zplus is very different from Wellman's qualitative probability
%networks~\cite{Wellman:90}, whose abstraction is based on direction of change, rather than order
%of magnitude relations ``at rest,'' as it were.
%Wellman's approach is more relevant when choosing interventions (as in his work
%on medical diagnosis and treatment), and less for fusion.

\hide{
The \assessor{} must combine the judgments of a wide variety of intrusion
detection systems (and potentially other relevant information sources), that use
widely varying sources of information and algorithms. Further, in general we
will not have access to the internals of these sensors.  In such an environment,
it is not realistic to expect good models of the response of these sensors; in
particular, exact measures of P(sensor response|event) are not available.  There
have been some attempts to investigate sensor response (e.g., the studies
conducted by Lincoln Labs~\cite{lippmann00:eval}), but the results seem heavily dependent on the
context in which the sensors are deployed.

The issue of prior probabilities also militates against the use of exact
probabilities.  In order to use an exact Bayesian method, we would need not only
the detection probability, $P(\mathit{sensor\ response}|\mathit{event})$, and
the false alarm probability, $P(\mathit{sensor\ response}|\neg \mathit{event})$,
but also $P(\mathit{event})$, a measure of the prior
probabilities of the events that interest us, in this case the attacks and the
benign events that can cause false positives.  Even in the most constrained
environments, the probabilities of the various attacks, are unlikely to be
available to us,  and the Scyllarus system is designed for application across a
wide variety of enterprises.  Further, the probability distributions for benign
events are likely to be of odd forms (e.g., one's own network-mapping software
runs at particular times of the day).  So our solution must tolerate vague
measures of likelihood.

Finally, in this domain, as with most practical applications of probabilistic
updating, the effect of the evidence will \emph{usually} overwhelm the effect of the prior
likelihoods(e.g., \cite{pradhan:96}). So inexactitude in the quantities specified will not matter to our
final conclusions.}

Under \zplus\ we may apply the normal operation
of probability theory
%% TODO Restore for QR'15
% --- conditionalization, Bayes' law, etc. ---
but the arithmetic operations we use for them change.  Rather than multiplying
probabilities, we add degrees of surprise%
%% TODO Restore for QR'15, but it's in the table anyway
%; rather than adding
%probabilities, we take the minimum surprise%
.  Goldszmidt and
Pearl~\shortcite{Goldszmidt:96} provide the following substitutions
between the quantitative probability function $P$ and its qualitative
counterpart $\kappa$ for formulas $\phi$ and $\psi$, and primitive events, $e$.
\begin{displaymath}
  \small
  \begin{array}{|l|l|}
    \hline
  P(\psi) = \sum_{e \in \psi} P(e) & \kappa(\psi) =
  \operatorname{min}_{e \in \psi} \kappa(e) \\ \hline
  P(\psi) + P(\neg \psi) = 1 & \kappa(\psi) = 0 \vee \kappa(\neg \psi) = 0
  \\ \hline
  P(\psi | \phi) = P(\psi \wedge \phi) / P(\phi) &
  \kappa(\psi | \phi) = \kappa(\psi \wedge \phi) - \kappa(\phi)
  \\ \hline
\end{array}
\end{displaymath}
\noindent
In a Bayes network, with conditionally independent $\omega$ and
$\phi$ we further have $\kappa(\omega \wedge \phi) = \kappa(\omega) + \kappa(\phi)$.
\hide{Instead of the probability
of an event being the sum of the probabilities of the primitive outcomes that
make up that event, the degree of surprise of an event is the minimum of the
degrees of surprise of the primitive outcomes that make it up. Instead of having
the probabilities of mutually exclusive and exhaustive events sum to one, at
least one of a set of mutually exclusive and exhaustive events must be
unsurprising. Finally, we have an analog of Bayes' law in which the normalizing
operation consists of subtraction rather than division.
}

\hide{We used Bayesian networks to help us in modeling and solving the correlation problem. Bayesian networks are 
ways of graphically capturing probabilistic reasoning. They are useful in expert systems because they 
simplify knowledge acquisition and, by capturing (conditional) independences, simplify computation 
(Pearl, 1988). In particular, in the domain of intrusion detection, Bayes nets help us capture several 
important patterns or probabilistic reasoning:
\begin{itemize}
\item	Reasoning based on evidence merging; 
\item	``Explaining away'' reports by alternative explanations.  E.g., if a benign event accounts for a number 
of reports, those reports will be explained away, and no longer provide support for more alarming 
hypotheses.

\item	Abstraction reasoning that employs the sub\-class/su\-per\-class relationships in the event dictionary.

\item	Part/whole reasoning, to recognize complex composite events.

\item	Distinguishing between judgments that are based on different sensor bases and those that use the same 
sensor.  This helps us distinguish between cases when two sensors provide support for each other and 
when we simply have redundant reports (e.g., two network intrusion detection systems using exactly 
the same algorithm that see the same traffic, at two different points).
\end{itemize}
A Bayesian network is a directed, acyclic graph (DAG) depicting a set of random variables. Edges between 
nodes in the DAG represent causal influences. Using a Bayesian network, we can capture a joint 
distribution factorized into unconditional probabilities for root nodes and conditional probability tables for 
non-root nodes. The conditional probability tables contain probability distributions for the child nodes, 
conditioned on all the values of their parents.
}

There are a number of efficient algorithms for finding the posterior distributions of Bayesian networks, 
conditional on observations of some of the random variables. These algorithms may readily be adapted to 
provide posterior $\kappa$ rankings instead of probabilities.
MIFD translates Bayes networks into an
%% DONE Charniak:88 is Charniak and Goldman --- withhold this one too? No --
%% it's not obviously tied to this paper, so we don't need to worry about it
%% breaching anonymity. [2014/08/24:rpg]
ATMS~\cite{Forbus+DeKleer:1993}; see
%% For anonymous:
%\cite{Charniak:88,Poole:93,provan-ijcai89}
%and \anoncite{goldman:09Scyllarus}
%% De-anonymized:
\cite{Charniak:88,Poole:93,provan-ijcai89,goldman:09Scyllarus}
%
for this encoding.
This implementation is not especially efficient, but we have found
it is I/O which dominates runtime, and we apply several optimizations to
handle common special cases.
%% TODO Restore for QR'15
%To date (somewhat to our disappointment!) optimizing the general \zplus
%inference method has not been key to system performance.
%
% To date, finding the optimal inference method
% % in terms of runtime
% has been
% less important than finding an algorithm easy to modify for \zplus.

% Extended this to address [[file:todo.txt::*Review%203:%20Limitation%20to%20three%20classes][Review 3: Limitation to three classes]]
\mifd's assessment ranks hypotheses as either
\emph{likely}, \emph{plausible} or \emph{unlikely}. A hypothesis $h$,
conditioned on evidence $\omega$,
is \emph{likely} if
$\kappa(h|\omega) < \kappa(\neg h|\omega)$, \emph{plausible} if $\kappa(h|\omega) = \kappa(\neg h|\omega)$
and \emph{unlikely} otherwise.
That is, a hypothesis is likely (unlikely) if some maximally likely scenario labels it as
true (false) , and no maximally likely scenario labels it as false (true).  A
hypothesis is plausible, in any other situation.
Note that this restriction to three classes applies to \mifd's output,
not its inputs --- the \tkappa\ values of \mifd's underlying model
range over arbitrary natural numbers as needed to distinguish
qualitatively distinct likelihoods.
It would be possible to alter the assessment process to provide more classes of
likelihood, using the difference in kappa ranks between the maximally likely
scenario in which a hypothesis is labeled as true (false) and the one in which
it is labeled as false (true).  We have not seen the need to do so and, indeed,
worry that additional precision would come at the cost of validity.

The MIFD system requires comparatively few \tkappa parameters given the above
design.  For sensors, we need \fpKappa, and for events (attack and benign)
we need $\kappa(\text{event})$.  In practice, we assign global defaults for
these, based on how generally accurate the input \idses are: for example, we set
these \tkappa\ values so that it takes $n = 2\ \text{or}\ 3$ sensors for us to judge an event
as \emph{likely}, with
the following consistency constraints:
\begin{align}
  \label{eqn:coherence-ineq}
  \kappa(\text{false-positive}),\kappa(\text{benign})
  &< \kappa(\text{attack}) \\
 &< n \cdot \kappa(\text{false-positive}) \nonumber
\end{align}
That is, a single sensor false positive is less surprising than an
  attack event;
a benign event is also less surprising than an attack; and
an attack is less surprising than false positives from $n$
  of the sensors detecting that event.
%
With a few exceptions, this paper assumes $n=3$ sensors, so we
have $0 < \kappa(\text{benign}) < \kappa(\text{attack}) 
    < 3 \kappa(\text{false-positive})$.
% \begin{displaymath}
%   \begin{split}
%     0 &< \kappa(\text{benign-event}) < \kappa(\text{attack-event}) \\
%     &< 3 \kappa(\text{false-positive})
%   \end{split}
% \end{displaymath}
% \noindent
In actual deployments, we then
nudge the false positive rankings up in the rare cases where we have a
particularly good sensor, or down if we have a particularly bad sensor.
%% TODO Restore for QR'15
%We are less likely to adjust attack event probabilities, but
%there are exceptions such as scanning (very common), and some complex events.


%%% Local Variables: 
%%% mode: latex
%%% TeX-master: "main"
%%% End: 

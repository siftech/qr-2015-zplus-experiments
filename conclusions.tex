
\section{Conclusions}
\label{sec:conclusions}

The results presented in this paper clarify why Scyllarus and
MIFD have been successful in practical deployments.
They show that the qualitative scheme they use is not sensitive to
the actual probabilities of events, and that its performance degrades
gracefully.

We would like to remind the reader that imprecise probabilities 
are forced upon us by the nature of the cyber intrusion detection domain: it's
not just a matter of not having the right machine learning technique.  The
true probabilities are non-stationary, they involve an adversarial process,
vary from location to location, and are difficult if not impossible to learn
because of the absence of valid labeled training data.
The use of a small number of qualitatively distinct likelihood levels also aids
us in the knowledge engineering process.

In practice, we also find that the results of \zplus calculations are easy to
understand.  We have found this through experience explaining the output of our
systems to users, and through experience debugging.  When we draw out the Bayes
networks, we can think of using assumptions to assemble the ``cheapest''
explanation for a set of observations.  This can be simpler to understand than
the exact computations in a Bayes net.

Our experimental results also justify the claim that by combining the output of
multiple sensors, even very noisy sensors, \ids fusion can tame the high false
positive rates that plague the field of intrusion detection.  These results are
somewhat independent from the question of the adequacy of the qualitative
calculus.  In a system that performed diagnosis/fusion using conventional
probability theory, it can be seen that multiple sensors that fail independently
will tend to perform well:  it's easier to drive the probability of error down
by multiplying failure probabilities ($p^n$) than to try to drive down a single
sensor's $p$ of failure.  However, our results show that substituting \zplus for
conventional probability theory preserves this desirable characteristic of
probabilistic reasoning. Our results also show resistance to a moderate degree
of correlation in sensor failures when using \zplus.

Since our results predict the circumstances under which IDS fusion will work and
will fail, they can also be used to inform the design and deployment of IDSes
for effective incorporation in a fusion system.
Finally, our results should encourage prospective users
of qualitative schemes based on probabilistic reasoning,
and promote deeper examination of % the usefulness and limits of
such systems.
In future work, we would like to examine more complex inference patterns that
arise from ``knotty'' Bayes nets, and the accuracy of the assessment process in
models that take into account how attacks spread through the topology of the underlying computer network.
\hide{In current work, we are extending our evaluation to consider MIFD's behavior
when topology is critical.  Cyber attacks
spread through network links, both actual
communications links and superimposed networks induced by protocols.
}


%%% Local Variables: 
%%% mode: latex
%%% TeX-master: "main"
%%% End: 

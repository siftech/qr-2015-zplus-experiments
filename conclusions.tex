
\section{Conclusions}
\label{sec:conclusions}

The results presented in this paper clarify why Scyllarus and
MIFD have been successful in practical deployments.
They show that the qualitative scheme they use is not sensitive to
the actual probabilities of events, and that its performance degrades
gracefully.
The results also justify the claim that by combining the output of multiple sensors, even
very noisy sensors, \ids fusion can tame the high false
positive rates that plague the field of intrusion detection.
Since our results predict the circumstances under which IDS fusion will work and
will fail, they can also be used to inform the design and deployment of IDSes
for effective incorporation in a fusion system.
Finally, our results should encourage prospective users
of qualitative schemes based on probabilistic reasoning,
and promote deeper examination of % the usefulness and limits of
such systems.
\hide{In current work, we are extending our evaluation to consider MIFD's behavior
when topology is critical.  Cyber attacks
spread through network links, both actual
communications links and superimposed networks induced by protocols.
}


%%% Local Variables: 
%%% mode: latex
%%% TeX-master: "main"
%%% End: 
